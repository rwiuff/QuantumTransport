%!TEX root = ../Main.tex
\begin{wrapfigure}[10]{r}{.4\textwidth}
	\vspace{-1em}
	\centering
	\includegraphics[width=.4\textwidth]{Figures/NPGintroGraphic.eps}
	\caption{Drawing of a nanoporous graphene device.}\label{introGraphic}
\end{wrapfigure}
In 2018, an article\cite{Moreno199}, published in Science presented a novel bottom-up approach to synthesise so called nanoporous graphene devices (NPGs). It proved to be ground breaking work for what, at present time, is a new exiting field of research in design of nano circuitry devices. These devices are made up of single layered graphene with periodic holes (hence the ``porous''). The remaining graphene constitutes ribbons called graphene nano ribbons (GNR) and bridges in the structures. \cref{introGraphic} shows how one such structure might look like.
Because of graphenes electrical properties\cite{calogero_electron_2019}, one should be able to finely control the electron currents in the devices and thus create nano meter circuits for use as e.g. chemical detectors. As a result of its novelty, fundamental electrical properties of the material has yet to be uncovered. Therefore, one must explore promising effects through theoretical simulation before fabrication of actual devices can be done.\newline
The aim of this project is to develop numerical tight-binding routines in Python using NumPy. Electron transport can then be simulated using non-equilibrium Green's functions as well as clever recursion algorithms. From here we obtain transmission and band structures for simulated nano devices.\newline
Generally speaking the community uses DFT-based simulations through tools like those from the SIESTA\cite{Soler_2002} project (TBtrans). Results are then analysed using SISL\cite{zerothi_sisl}. The DFT results can then be extrapolated to larger scales\cite{calogero_electron_2019}. However DFT programs are complicated and might seem as a black box. To get a better understanding of electron transport, we avoid DFT and rely solely on tight-binding simulations. For simplicity we take only into account the nearest neighbouring atoms, and we consider only planar structures.  We then confirm the validity of the developed products by comparing result to those from SIESTA.\newline
To summarise:
\begin{enumerate}
	\item Apply quantum mechanics for electron transport in graphene based devices.
	\item Develop numerical methods (using recursion algorithms, linear algebra) with NumPy to implement tight-binding.
	\item Calculate band structures and transmission plots for various devices.
	\item Gather single-particle Green’s functions and LDOS of said devices.
	\item Compare the obtained results and discuss whether or not they sufficiently resemble DFT based simulations.
\end{enumerate}
The report is organised on the following way:
\begin{enumerate}
	\item \cref{theorysec,hamilsec,greensec,transec} deals with the development of the scripts. First by introduction of basic theoretical concepts, followed by how these concepts are implemented practically through programming.
	\item \cref{testsec} deals with the generated results of calculations on various NPGs and the comparison with DFT calculations done on the same systems.
	\item \cref{Conc} summarises the results and concludes the project.
\end{enumerate}
The code repository (which also includes the \latex files for this report) can be found on GitHub: \faGithub \ \url{https://github.com/rwiuff/QuantumTransport}