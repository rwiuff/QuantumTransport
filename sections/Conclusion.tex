%!TEX root = ../Main.tex
Concluding on section \cref{theorysec,hamilsec,greensec,transec}, simple numerical methods have been developed and implemented with success. The resulting LDOS, band and transmission plots, especially for NPG, show that the method indeed is capable of reproducing results from DFT and TBtrans calculation. However, there are small discrepancy in the results obtained when comparing with DFT. The reasons for the discrepancy is not immediately clear from the results but there might be a couple of things to point out and keep in mind. The method has been developed on the assumption that all atoms are in the plane and they, as a baseline all have the same potential. This might cause some of the small differences that one can see in the results, especially the transmission plots of \cref{transmissionplots}. So the DFT approach might have picked up some these effects that the developed method could not.\\
In \cref{testsec} the program showed that it too could reproduce results of much more complicated systems with multiple atom species, not just carbon. The initial results where not strongly resembling the DFT calculations, but after some tweaking of specific on-site potentials, the agreement became very good. This gave some valuable insights as to what happens chemically in NPG when it is functionalised with different atoms. These discoveries also made it more clear which flaws the method had and how to go about simulating even more complicated systems. One approach which should be implemented is automating the manipulation of the on-site potentials of species bonded to the bridges in NPG. The scripts used in this project only took into account the specific atoms bonded to the NPG and changed their potential. However, as one can see in potential maps in the figures of \cref{testsec}, the potential is not only changing at the specific sites where the atoms have bonded. It also changes the on-site potential of all the other atoms in the system. This was not accounted for in the developed method. However, there is a system as to how these potentials change. In future work, a possible way to account for the potential change, is to look at each carbon atom in the system, and count how many bonds it makes to other atoms. The amount of bonds the atom makes is, in part, determining its potential. By changing the potential relative to how many bonds they are making to other atoms, it would be possible to get a more precise picture of the potentials in the system and thus give an even more precise result when calculating band plots and transmission. \\
This proposal is for working out the details of producing the most accurate results possible. The main fact is that the project succeeded in making a program that could reproduce result that qualitatively was the same as the DFT, but in a much simpler way. This should pave the way for a method faster computing of all different kinds of NPG systems in a field that is still evolving rapidly. By showing a method where DFT has been avoided, it also gives value to the intuitive understanding of NPG systems which should make this field more available to a broader audience.  