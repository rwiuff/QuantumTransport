In this section are range of tests will be conducted on different NPG structures in order to uncover the effect of manipulation of the bridges in the NPG. Namely manipulations will be done on two different kinds of bridge structures. The \textit{meta} and \textit{para} bridges. In \cref{} the difference between the structures can be seen in relation to the bridge in the normal NPG. The bridges will be manipulated by insertion of either Flour or Oxygen atoms or OH-groups. The atoms/molecules will be inserted on each side of the nano-ribbon either in a symmetric or asymmetric fashion (see \cref{}). In \cref{table} a schematic overview of the different tests can be seen. For preliminary tests the script developed and described in the previous sections will be used. For more elaborate tests, a transition to \textit{sisl} will be made. \subsection{Differences in para and meta bridges}
In broad terms the difference in the meta and para structures lies in the path an electron will travel to get across the bridge. In the para bridge, the path across the aromatic ring is symmetric and so the electron will pass above or below with equal probability. Since the para bridge has three bonds in each direction across the ring, the path length in the para bridge is the same on each side (see\cref{}). This will cause constructive interference once the waves meet on the other side of the ring, which results in spreading in the density of states from one nano-ribbon to the next. For the meta bridge the way across the aromatic ring is not symmetric in the sense that there is two bonds across the path below and four bonds across on the path above (see \cref{}). This will cause a shift by half a wavelength between the two paths and thus create destructive interference between waves meeting on the other side of the ring. The effect of this is confinement in the density of states in the ribbons. 
\textit{the following sections are written from the notes taken during Isaacs presentation and might need editing as the actual tests are completed(altered)}
\subsection{Tests with insertion of flour atoms}
Tests by insertion of flour atoms will be done for both para and meta bridges. Each  of those will be tested with symmetric and asymmetric placement of flour atoms. The configuration of electrons in flour is as such the it does not contribute with any extra pi-electrons to the system and thus adding an extra flour atom to the bridge will still result in an electrostatic system (before injection of an electron). One of the goals of the test is to see whether the effect of adding a flour atom will be global (in relation to the NPG-system) or more localised. Another is to see whether the on-site Hamiltonian along with its hopping matrices will change by addition of a flour atom. 
\subsection{Tests with insertion of oxygen atoms}
In the same manner as the test with flour, oxygen atoms will be added to the bridges for the second test. The notable difference being that oxygen will add an extra pi-electron to the system. This means that one has to consider the electrostatic effects as well as additional effects from the added pi-electron. Practically, and as an initial test, this is done by adding an extra carbon atom as a "dummy" for the oxygen atom. Following that one might have to change specific on-site values (the place where the atom is inserted as well as surrounding atoms) to simulate the addition of the oxygen atom. The goal here is to check difference between just adding an extra oxygen (carbon) atom and changing the on-site and hopping values. 
\subsection{Tests with insertion of OH-groups}
The last case of inserting a OH-group should intuitively take one back to a situation similar to that of adding a flour atom. Since the extra hydrogen atom in the OH+group will remove the extra pi-electron from the equation, one would expect a electrostatic case, just as with flour. 