%!TEX root = ../Main.tex
\subsection{Ballistic quantum transport}
As graphene is two dimensional material that consists of carbon atoms arranged in a hexagonal pattern, features in such a material can approach nanometer and sub nanometer scales. Because of the small scale the electrical properties and the electrical nature of the material is greatly changed. Normal drift-diffusion current models describe electric charges per area and current per area, but because the conductor is graphene, it can be considered one dimensional. This makes drift-diffusion models insufficient to describe the electrical transport and properties of graphene because they are based on scattering of multiple electrons and the mean free path between scattering. In graphene only a few electrons at a time are considered when modeling electron transport and it is therefore necessary to use quantum mechanics to describe the transport of electrons in the material.
\subsection{\mathinhead{\pi}{\pi}-orbitals and \mathinhead{\pi}{\pi}-electrons}
\begin{wrapfigure}[7]{r}{.3\textwidth}
	\vspace{-2.3em}
	\centering
	\begin{tikzpicture}
		\chemfig{C*6(-C-C-C-C-C-)}
	\end{tikzpicture}
	\caption{Graphene lattices consists of hexagonal arrangements of carbon atoms.}\label{ring}
\end{wrapfigure}
The main scope of this paper is dealing with electron transport in novel nanoporous graphene devices.
When modeling such transport one needs to adress the orbital structure of carbon lattices and later this will motivate the use of tightbinding and Green's functions.
In its basic form graphene can be devided into rings of carbon atoms as shown in \cref{ring}. In the (\(x,y\))-plane the carbon atoms are bound in \(sp^2\) orbitals as shown in \cref{sp2}.
\begin{figure}[H]
	\centering
	\resizebox{.4\textwidth}{!}{
		\begin{tikzpicture}
			\satom[name=C, color=blue, pos={(0,0)}]{
				blue/60/north east/2/1,
				blue/180/west/1,
				blue/300/south east/2/1
			}
			\satom[name=C, color=blue, pos={(1,1.4)}]{
				blue/0/east/2/1,
				blue/120/north west/1,
				blue/240/south west/2/1
			}
			\satom[name=C, color=blue, pos={(2.74,1.4)}]{
				blue/60/north east/1,
				blue/180/west/2/1,
				blue/300/south east/2/1
			}
			\satom[name=C, color=blue, pos={(3.74,0)}]{
				blue/0/east/1,
				blue/120/north west/2/1,
				blue/240/south west/2/1
			}
			\satom[name=C, color=blue, pos={(2.74,-1.4)}]{
				blue/60/north east/2/1,
				blue/180/west/2/1,
				blue/300/south east/1
			}
			\satom[name=C, color=blue, pos={(1,-1.4)}]{
				blue/0/east/2/1,
				blue/120/north west/2/1,
				blue/240/south west/1
			}
		\end{tikzpicture}}
	\caption{Carbon atoms in a hexagonal lattice are \(sp^2\) hybradised in the (\(x,y\))-plane.}\label{sp2}
\end{figure}
This hybradisation lock all but one valence electron for the carbon atoms. These electrons exists in a p-orbital in the \(z\)-direction.
\cref{p} shows the valence orbitals of carbon.
\begin{figure}[H]
	\begin{center}
		\begin{tikzpicture}
			\orbital[pos = {(0,3)}] {s}
			\node[above] at (0,4) {s};
			\orbital[pos = {(2,3)}]{px}
			\node[above] at (2,4) {p$_x$};
			\orbital[pos = {(4,3)}]{py}
			\node[above] at (4,4) {p$_y$};
			\orbital[pos = {(6,3)}]{pz}
			\node[above] at (6,4) {p$_z$};
		\end{tikzpicture}
		\caption{The valence orbitals of carbon.}
		\label{p}
	\end{center}
\end{figure}
The last electron in the p\(_z\) orbital does not mix with the tightly bound s, p\(_x\) and p\(_y\) electrons and moves more freely. Thus these electrons have higher energies compared to the \(sp^2\) electrons and occupy states at the Fermi level. These electrons dominates transport in the graphene lattice. The p\(_z\) orbital is also known as the \(\pi\)-orbital and as such the electron lying there is called a \(\pi\)-electron. Through a carbon lattice the \(\pi\)-electrons will travel through \(\pi\)-orbitals, switching sign as they go as shown in \cref{sign}.
\begin{figure}[H]
	\begin{center}
		\pgfdeclarelayer{background}
		\pgfdeclarelayer{middle}
		\pgfdeclarelayer{foreground}
		\pgfsetlayers{background,middle,main,foreground}
		\begin{tikzpicture}
			\begin{pgfonlayer}{background}
				\orbital[pos = {(6,6)}]{-pz}
				\node[above] at (6,7) {-p$_\pi$};
				\orbital[pos = {(4,6)}]{pz}
				\node[above] at (4,7) {p$_\pi$};
				\draw[dashed, very thick] (6,6) -- (4,6);
				\draw[dashed, very thick] (7,4.73) -- (6,6);
				\draw[dashed, very thick] (4,6) -- (3,4.73);
			\end{pgfonlayer}
				\orbital[pos = {(7,4.73)}]{pz}
				\node[above] at (7,5.73) {p$_\pi$};
				\orbital[pos = {(3,4.73)}]{-pz}
				\node[above] at (3,5.73) {-p$_\pi$};
			\begin{pgfonlayer}{foreground}
				\orbital[pos = {(4,3.46)}]{pz}
				\node[above] at (4,4.46) {p$_\pi$};
				\orbital[pos = {(6,3.46)}]{-pz}
				\node[above] at (6,4.46) {-p$_\pi$};
				\draw[dashed, very thick] (4,3.46) -- (6,3.46);
			\end{pgfonlayer}
			\draw[dashed, very thick] (6,3.46) -- (7,4.73);
			\draw[dashed, very thick] (3,4.73) -- (4,3.46);
		\end{tikzpicture}
		\caption{When going from one carbon atom to another, the \(\pi\)-electron goes between p\(_\pi\) and -p\(_\pi\).}
		\label{sign}
	\end{center}
\end{figure}
\subsection{Tight-binding}
Now that the transport carrying electrons are defined, one must choose a formalism for the transport itself. Introducing: \textbf{``The Tight-Binding approximation''}.
In this approximation the electrons are considered being tightly bound to the atoms. Contrary to a free electron gass approximation, the electrons does not spend time in between orbitals, but jump from orbital in atom \(a\) to orbital in atom \(b\). In this world view the Hamiltonian operator is a matrix of hopping elements for a collection of neighbouring atomic orbitals, i.e. molecular orbitals. This can be done by describing the orbitals as a Linear Combination of Atomic Orbitals (LCAO). The solution to the Schrödinger equation is then:
\begin{align}
	\Psi_{\mathrm{MO}} = \sum_{\alpha,R}c_{\alpha,R}\phi_{\alpha}(R)
\end{align}
where \(\phi_{\alpha}(R)\) is some atomic orbital at position \(R\), with \(\alpha\) denoting the valence of the orbital (\(2s,2p_x,2p_y,2p_z\)). In electron transport the states close to the Fermi level is of interest. These are namely the highest occupied moelcular orbitals (HOMO), or the lowest unoccupied molecular orbitals (LUMO).
